\documentclass[letterpaper,12pt]{article}

\usepackage{amssymb,amsmath}

\begin{document}
\begin{table}[ht]
\caption{Nonlinear Model Results} % title of Table
\centering  % used for centering table
\begin{tabular}{c c c c} % centered columns (4 columns)
\hline\hline                        %inserts double horizontal lines
Case & Method\#1 & Method\#2 & Method\#3 \\ [0.5ex] % inserts table 
%heading
\hline                  % inserts single horizontal line
1 & 50 & 837 & 970  \\ % inserting body of the table
2 & 47 & 877 & 230  \\
3 & 31 & 25  & 415  \\
4 & 35 & 144 & 2356 \\
5 & 45 & 300 & 556 \\ [1ex]      % [1ex] adds vertical space
\hline %inserts single line
\end{tabular}
\label{table:nonlin} % is used to refer this table in the text
\end{table}

\begin{equation}\label{xx}
\begin{split}
a& =b+c-d\\
& \quad +e-f\\
& =g+h\\
& =i
\end{split}
\end{equation}

The table above, \ref{xx}, is an example of a floating body in \LaTeX!

\end{document}